\documentclass[a4paper, 12pt]{article} % Tipo de documento, formato do papel e tamanho da fonte

% Codificação de caracteres e fonte
\usepackage[utf8]{inputenc}
\usepackage[T1]{fontenc}

% Configurações do Português
\usepackage[portuguese]{babel}
\usepackage{hyphenat}
\hyphenation{mate-mática recu-perar}

% Configurar margens, espaçamento e parágrafos
\usepackage[left=3cm, top=3cm, right=2cm, bottom=2cm]{geometry}
\usepackage{setspace} 
\setlength{\parindent}{1.25cm}
\usepackage{indentfirst}

% Incluir imagens
\usepackage{graphicx} 
\usepackage{float}

% Gerenciar bibliografia
% \usepackage[backend=biber]{biblatex}
% \addbibresource{referencias.bib}
% \usepackage{csquotes}

% Referências clicáveis
% \usepackage{hyperref}

% Funcionalidades adicionais na criação de tabelas
% \usepackage{array} % tamanho fixo para colunas ou tabelas inteiras
% \usepackage{tabularx} % ambiente tabular mais flexível
% \usepackage{multirow} % união de linhas e colunas para criar células maiores
% \usepackage{longtable} % tabelas que ocupam mais de uma página
% \usepackage[table]{xcolor} % inserção de cores na tabela

% Lista ordenada com letras
% \usepackage[shortlabels]{enumitem}

\begin{document}

% Capa do trabalho
\begin{center}

  \includegraphics[scale=1]{logo-infnet.png}\\
  \large
  \textbf{ESTI - ESCOLA SUPERIOR DA TECNOLOGIA DA INFORMAÇÃO}\\
  \textbf{GRADUAÇÃO EM XXXX}\\
  \textbf{BLOCO DE XXX}
  
  \vspace{6.5cm}
  NOME DO(A) ESTUDANTE \\
  NOME DO(A) PROFESSOR(A)

  \vspace{4.5cm}
  \textbf{TÍTULO DO TRABALHO}\\
  \textbf{DISCIPLINA}

  \vspace{3cm}
  Rio de Janeiro \\
  DIA de MÊS de 20xx.

  % Remover numeração na página da capa
  \thispagestyle{empty}

\end{center}
\newpage

% Início do trabalho
\onehalfspacing

\end{document}
